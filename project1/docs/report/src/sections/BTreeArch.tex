\section{Formal Definition of Our B Plus Tree}
\label{appendix:bplustreearch}

Our \bplustree is constructed following the foundational definition of B-trees as given in \cite{cormen2022introduction}, with key modifications to accommodate the structure and search needs of our data storage and indexing system.

The primary difference between our \bplustree and the one in \cite{cormen2022introduction} is that in our implementation is that only the leaf nodes store the actual data records, while internal nodes store only the keys.

\subsection{Node Structure}

Each node $x$ in the \bplustree has the following properties:\begin{enumerate}
        \item \textbf{Number of Keys:} $n_x$, representing the number of keys currently stored in node $x$. For internal nodes, these keys help guide the search by partitioning the search space.
        \item \textbf{Keys:} The $n_x$ keys, $k^{(1)}_x,k^{(2)}_x,\cdots,k^{(n_x)}_x$, stored in monotonically increasing order, such that $k^{(1)}_x<k^{(2)}_x<\cdots<k^{(n_x)}_x$,
        \item \textbf{Leaf Indicator:} $\ell_x$, a boolean value that is \texttt{true} if $x$ is a leaf and \texttt{false} if $x$ is an internal node.
        \item \textbf{Pointers:} Each node, depending on weather it is an internal or leaf node, contains a specific number of pointers (described in the next sections) that link to either child nodes or data blocks
    \end{enumerate}

\subsection{Internal Nodes}

Each internal node $x$ contains $n_x + 1$ pointers, denoted  $s^{(1)}_x,s^{(2)}_x,\cdots,s^{(n_x + 1)}_x$, which point to its child nodes. These pointers separate the key space such that each pointer directs the search towards the subtree containing the relevant keys.
        \begin{itemize}
            \item \textbf{Pointers as Navigational Aids}: The internal nodes do not store actual data records but only serve as guides for navigating to the correct leaf node or subtree where the data is stored.
        \end{itemize}
        \begin{itemize}
            \item \textbf{Key-based Navigation}: The keys in an internal node are used to define search boundaries. If a search for a key \textit{k} lands in node \textit{x}, then the key is passed to the appropriate child pointer based on the relationship: $$k_1<k_x^{(1)}\le k_2<k_x^{(2)}\le \cdots <k_x^{(n_x)}\le k_{n_x+1}$$
where $s^{(i)}_x$ leads to the subtree containing all keys in the range defined by the keys in the parent node.
        \end{itemize}

\subsection{Leaf Nodes}

Each leaf node $x$ contains $n_x + 1$ pointers, denoted $s^{(1)}_x,s^{(2)}_x,\cdots,s^{(n_x + 1)}_x$, which are structured differently from the internal node pointers:

        \begin{enumerate}
            \item For $1\le i\le n_x$,each pointer $s^{(i)}_x$ points directly to the data record corresponding to the key $k_x^{(i)}$.
            \item The last pointer, $s^{(n_x+1)}_x$ points to the \textbf{next leaf node} in the \bplustree's depth-first search (DFS) order. This allows for efficient range queries by following the chain of leaf nodes:\begin{itemize}
            \item If $x$ is the last leaf node, then $s^{(n_x+1)}$ is $\emptyset$, indicating the end of the tree.\end{itemize}
        \end{enumerate}


\subsection{Key Separation in Subtrees}

The keys $k_x^{(i)}$ separate the ranges of keys stored in each subtree. For any key subtree, $k_i$ is stored in the subtree with root $s_x^{(i)}$, the following relationship holds:  $$k_1<k_x^{(1)}\le k_2<k_x^{(2)}\le \cdots <k_x^{(n_x)}\le k_{n_x+1}$$

This property ensures that a search operation is guided down the correct subtree by comparing the key against the keys in the parent node.

\subsection{Leaf Depth}
All leaf nodes in the \bplustree are located at the same depth, equal to the height $h_T$ of the tree. This ensures balanced access times for all records, as every search or insertion operation requires traversing the same number of levels to reach a leaf node.

\subsection{Key Boundaries and Node Capacity}
Each node in the \bplustree has lower and upper bounds on the number of keys it can hold, based on the integer parameter $t\ge 2$:
\begin{enumerate}
    \item Lower Bound:
    \begin{itemize}
        \item Any node(except the root) must contain at least $t-1$ keys.
        \item Each internal node, therefore, has at least $t$ children.
        \item The root node may contain fewer than $t-1$ keys but must contain at least one key if the tree is non-empty.
    \end{itemize}

    \item Upper Bound:
    \begin{itemize}
        \item Any node can hold a maximum of $2t-1$.
        \item An internal node can have up to $2t$ children, allowing for balanced growth as new records are inserted.
    \end{itemize}

\end{enumerate}

\subsection{Operations on the B+ Tree}

\begin{itemize}
    \item \textbf{Search:} To locate a record, the search process navigates from the root to the appropriate leaf node, using the keys in internal nodes to direct the search path. Once the correct leaf node is located, the corresponding data is fetched via the leaf node’s pointers.
    \item \textbf{Insertion:} Insertion occurs by first finding the correct leaf node for the key. If the leaf node has space for additional keys, the new key and data record are added. If the leaf is full, it is split, and a new key is promoted to the parent node. If the parent node also becomes full, the splitting and promotion process continues recursively, potentially resulting in the tree growing in height.
    \item \textbf{Range Queries:} Range queries are highly efficient in the \bplustree due to the linked list structure of the leaf nodes. Once the range's starting key is located, the query can traverse the linked list of leaf nodes to retrieve all records within the specified range.
\end{itemize}
